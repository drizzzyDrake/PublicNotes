% Options for packages loaded elsewhere
\PassOptionsToPackage{unicode}{hyperref}
\PassOptionsToPackage{hyphens}{url}
\documentclass[
]{article}
\usepackage{xcolor}
\usepackage{amsmath,amssymb}
\setcounter{secnumdepth}{-\maxdimen} % remove section numbering
\usepackage{iftex}
\ifPDFTeX
  \usepackage[T1]{fontenc}
  \usepackage[utf8]{inputenc}
  \usepackage{textcomp} % provide euro and other symbols
\else % if luatex or xetex
  \usepackage{unicode-math} % this also loads fontspec
  \defaultfontfeatures{Scale=MatchLowercase}
  \defaultfontfeatures[\rmfamily]{Ligatures=TeX,Scale=1}
\fi
\usepackage{lmodern}
\ifPDFTeX\else
  % xetex/luatex font selection
\fi
% Use upquote if available, for straight quotes in verbatim environments
\IfFileExists{upquote.sty}{\usepackage{upquote}}{}
\IfFileExists{microtype.sty}{% use microtype if available
  \usepackage[]{microtype}
  \UseMicrotypeSet[protrusion]{basicmath} % disable protrusion for tt fonts
}{}
\makeatletter
\@ifundefined{KOMAClassName}{% if non-KOMA class
  \IfFileExists{parskip.sty}{%
    \usepackage{parskip}
  }{% else
    \setlength{\parindent}{0pt}
    \setlength{\parskip}{6pt plus 2pt minus 1pt}}
}{% if KOMA class
  \KOMAoptions{parskip=half}}
\makeatother
\usepackage{longtable,booktabs,array}
\usepackage{calc} % for calculating minipage widths
% Correct order of tables after \paragraph or \subparagraph
\usepackage{etoolbox}
\makeatletter
\patchcmd\longtable{\par}{\if@noskipsec\mbox{}\fi\par}{}{}
\makeatother
% Allow footnotes in longtable head/foot
\IfFileExists{footnotehyper.sty}{\usepackage{footnotehyper}}{\usepackage{footnote}}
\makesavenoteenv{longtable}
\setlength{\emergencystretch}{3em} % prevent overfull lines
\providecommand{\tightlist}{%
  \setlength{\itemsep}{0pt}\setlength{\parskip}{0pt}}
\usepackage{bookmark}
\IfFileExists{xurl.sty}{\usepackage{xurl}}{} % add URL line breaks if available
\urlstyle{same}
\hypersetup{
  hidelinks,
  pdfcreator={LaTeX via pandoc}}

\author{}
\date{}

\begin{document}

In un'architettura \textbf{RISC-V}, la \textbf{CPU} è composta da
diverse \textbf{unità funzionali} che collaborano per eseguire
istruzioni. Ecco una panoramica delle principali unità:

\begin{longtable}[]{@{}
  >{\raggedright\arraybackslash}p{(\linewidth - 0\tabcolsep) * \real{0.0556}}@{}}
\toprule\noalign{}
\begin{minipage}[b]{\linewidth}\raggedright
\#\#\# \textbf{CU (Control Unit)}
\end{minipage} \\
\midrule\noalign{}
\endhead
\bottomrule\noalign{}
\endlastfoot
\#\#\# \textbf{ALU (Arithmetic Logic Unit)} \\
📌 \textbf{Ruolo:} Esegue operazioni aritmetiche e logiche. 🔹
Addizione, sottrazione (\texttt{add}, \texttt{sub}) 🔹 Operazioni
logiche (\texttt{and}, \texttt{or}, \texttt{xor}) 🔹 Shift e rotazioni
(\texttt{sll}, \texttt{srl}, \texttt{sra}) 🔹 Confronti (\texttt{slt},
\texttt{sltu}) \^{}08fb26 \\
\end{longtable}

\subsubsection{\texorpdfstring{\textbf{FPU (Floating Point
Unit)}}{FPU (Floating Point Unit)}}\label{fpu-floating-point-unit}

📌 \textbf{Ruolo:} Esegue operazioni su numeri in virgola mobile.\\
🔹 Somma, sottrazione, moltiplicazione, divisione (\texttt{fadd.s},
\texttt{fmul.s}).\\
🔹 Conversioni tra interi e float (\texttt{fcvt.s.w}).\\
🔹 Comparazioni (\texttt{flt.s}, \texttt{feq.s}). \^{}3c2941

\begin{longtable}[]{@{}
  >{\raggedright\arraybackslash}p{(\linewidth - 0\tabcolsep) * \real{0.0556}}@{}}
\toprule\noalign{}
\begin{minipage}[b]{\linewidth}\raggedright
\#\#\# \textbf{IFU (Instruction Fetch Unit)}
\end{minipage} \\
\midrule\noalign{}
\endhead
\bottomrule\noalign{}
\endlastfoot
\#\#\# \textbf{IDU (Instruction Decode Unit)} \\
📌 \textbf{Ruolo:} Interpreta l'istruzione e attiva le unità necessarie.
🔹 Identifica il tipo di istruzione (aritmetica, memoria, branch, FPU).
🔹 Controlla quali registri sono coinvolti. 🔹 Imposta segnali per la
\textbf{ALU, FPU, Load/Store Unit}. \^{}f06d0e \\
\end{longtable}

\subsubsection{\texorpdfstring{\textbf{LSU (Load/Store
Unit)}}{LSU (Load/Store Unit)}}\label{lsu-loadstore-unit}

📌 \textbf{Ruolo:} Gestisce il trasferimento di dati tra registri e
memoria.\\
🔹 Caricamento dati (\texttt{lw}, \texttt{lb}, \texttt{ld}).\\
🔹 Memorizzazione dati (\texttt{sw}, \texttt{sb}, \texttt{sd}).\\
🔹 Accesso alla memoria cache. \^{}9f277d

\begin{longtable}[]{@{}
  >{\raggedright\arraybackslash}p{(\linewidth - 0\tabcolsep) * \real{0.0556}}@{}}
\toprule\noalign{}
\begin{minipage}[b]{\linewidth}\raggedright
\#\#\# \textbf{BU (Branch Unit)}
\end{minipage} \\
\midrule\noalign{}
\endhead
\bottomrule\noalign{}
\endlastfoot
\#\#\# \textbf{CSRU (Control and Status Registers Unit)} \\
📌 \textbf{Ruolo:} Gestisce registri speciali della CPU. 🔹 Controlla
eccezioni e interrupt (\texttt{mstatus}, \texttt{mepc}). 🔹 Gestisce la
modalità privilegiata (\texttt{mcause}, \texttt{mie}). 🔹 Abilita la FPU
e altre estensioni (\texttt{misa}). \\
\end{longtable}

\subsubsection{**MMU (Memory Management
Unit)}\label{mmu-memory-management-unit}

📌 \textbf{Ruolo:} Controlla l'accesso alla memoria principale e
virtuale.\\
🔹 Gestisce la \textbf{memoria virtuale} e la traduzione degli indirizzi
virtuali in fisici. 🔹 Controlla i permessi di accesso alla memoria.
\^{}1851aa

\begin{center}\rule{0.5\linewidth}{0.5pt}\end{center}

\end{document}
